\documentclass[12pt]{article}

% Paquetes usuales
\usepackage[spanish]{babel}
\usepackage{amsmath}
\usepackage{amsthm}
\usepackage{amsfonts}
\usepackage{amssymb}
\usepackage[a4paper, total={6.5in, 10in}]{geometry}
\usepackage{bm}
\usepackage{mathrsfs}
\usepackage{dsfont}
\usepackage{bm}
\usepackage{enumitem}

\usepackage{mdframed}
\usepackage{lipsum}

\newmdtheoremenv{lema}{Lema}

% Mis comandos
\newcommand{\RR}{\rm I\!R}
\newcommand{\NN}{{\rm I\!N}}
\newcommand{\CC}{\mathds{C}}
\newcommand{\QQ}{${\mathds{Q}}$}
\newcommand{\KK}{{\rm I\!K}}
\newcommand{\ZZ}{$\mathds{Z}$}

\newcommand{\abs}[1]{\left|#1\right|}
\newcommand{\norm}[1]{\lVert #1\rVert}

% Entornos
\theoremstyle{definition}
\newtheorem{theorem}[subsection]{Teorema}
\newtheorem{proposition}[subsection]{Proposición}
\newtheorem{lemma}[subsection]{Lema}
\newtheorem{corollary}[subsection]{Corolario}
\newtheorem{observation}[subsection]{Observación}

\newenvironment{ejemplo}{\noindent\textbf{Ejemplo: }}{}
\newenvironment{demostracion}{\noindent\textit{Demostración: } $ $ \newline}{\newline\qed}

\title{\textbf{Cuestiones práctica 2\\ {\small Ampliación de análisis numérico}}}
\date{}
\author{}

\begin{document}
\maketitle
	
\noindent\textbf{Cuestión 1:}
\begin{center}
	\begin{tabular}{|c || c | c  || c |}
		\hline
		\textbf{Familia} & \textbf{No. de condición} & \textbf{No. de celda} & \textbf{No. de experimentos} \\ 
		\hline
		\textbf{a)} & 9.8519e+08 & 2 & 652 \\
		\hline 
		\textbf{b)} & 32 & 5 & 409\\
		\hline
		\textbf{c)} & 3.3987e+05 & 3 & 425\\
		\hline 
		\textbf{d)} & 5.6914 & 8 & 233\\
		\hline
		\textbf{e)} & 448 & 3 & 506 \\
		\hline
	\end{tabular}
\end{center}
\noindent\textbf{Cuestión 2:}
El punto de vista estadístico que adoptamos es ver como respetan las cotas las diferentes matrices. De mejor a peor serían
\begin{enumerate}
	\item Matriz de Househölder.
	\item Matriz tridiagonal.
	\item Matriz de Vandermonde.
	\item Matriz triangular superior.
	\item Matriz de Hilbert.
El número de condición de la matriz explica como se comporta un sistema de ecuaciones sobre una perturbación en los datos. A menor número de condición menor es el cambio de los resultados. Como podemos observar la dos matrices con menor número de condición se comportan bien en las cotas.
\end{enumerate}
\noindent\textbf{Cuestión 3:} 
\begin{itemize}
	\item La instrución \texttt{L=log(mu);}
\end{itemize}
\end{document}