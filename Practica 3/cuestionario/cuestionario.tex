\documentclass[12pt]{article}

% Paquetes usuales
\usepackage[spanish]{babel}
\usepackage{amsmath}
\usepackage{amsthm}
\usepackage{amsfonts}
\usepackage{amssymb}
\usepackage[a4paper, total={6.5in, 10in}]{geometry}
\usepackage{bm}
\usepackage{mathrsfs}
\usepackage{dsfont}
\usepackage{bm}
\usepackage{enumitem}
\usepackage{graphicx}
\usepackage{subcaption}

\usepackage{mdframed}
\usepackage{lipsum}

\newmdtheoremenv{lema}{Lema}

% Mis comandos
\newcommand{\RR}{\rm I\!R}
\newcommand{\NN}{{\rm I\!N}}
\newcommand{\CC}{\mathds{C}}
\newcommand{\QQ}{${\mathds{Q}}$}
\newcommand{\KK}{{\rm I\!K}}
\newcommand{\ZZ}{$\mathds{Z}$}

\newcommand{\abs}[1]{\left|#1\right|}
\newcommand{\norm}[1]{\lVert #1\rVert}

% Entornos
\theoremstyle{definition}
\newtheorem{theorem}[subsection]{Teorema}
\newtheorem{proposition}[subsection]{Proposición}
\newtheorem{lemma}[subsection]{Lema}
\newtheorem{corollary}[subsection]{Corolario}
\newtheorem{observation}[subsection]{Observación}

\newenvironment{ejemplo}{\noindent\textbf{Ejemplo: }}{}
\newenvironment{demostracion}{\noindent\textit{Demostración: } $ $ \newline}{\newline\qed}

\title{\textbf{Cuestiones práctica 2\\ {\small Ampliación de análisis numérico}}}
\date{}
\author{}

\begin{document}
	\maketitle
	
	\noindent\textbf{Cuestión 1:} En la instrucción \texttt{G = chol(A, 'lower')} el parámetro \texttt{'lower'} indica a la función que utilice la parte triangular inferior de la matriz  y la diagonal de $A$ para calcular la factorización de Cholesky. Si realizamos el experimento sin utilizar el parámetro \texttt{'lower'} observamos dos sucesos:
	\begin{enumerate}
		\item Si no especificamos que la matriz es dispersa el cálculo de la factorización es más lento sin dicho argumento para $n < 10^2$ aproximadamente pero a partir de aquí los tiempos son prácticamente iguales.
	\end{enumerate}

	\noindent\textbf{Cuestión 2:}
\end{document}