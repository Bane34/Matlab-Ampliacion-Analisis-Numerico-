\documentclass[12pt]{article}

% Paquetes usuales
\usepackage[spanish]{babel}
\usepackage{amsmath}
\usepackage{amsthm}
\usepackage{amsfonts}
\usepackage{amssymb}
\usepackage[a4paper, total={6.5in, 10in}]{geometry}
\usepackage{bm}
\usepackage{mathrsfs}
\usepackage{dsfont}
\usepackage{bm}
\usepackage{enumitem}
\usepackage{graphicx}
\usepackage{subcaption}

\usepackage{mdframed}
\usepackage{lipsum}

\newmdtheoremenv{lema}{Lema}

% Mis comandos
\newcommand{\RR}{\rm I\!R}
\newcommand{\NN}{{\rm I\!N}}
\newcommand{\CC}{\mathds{C}}
\newcommand{\QQ}{${\mathds{Q}}$}
\newcommand{\KK}{{\rm I\!K}}
\newcommand{\ZZ}{$\mathds{Z}$}

\newcommand{\abs}[1]{\left|#1\right|}
\newcommand{\norm}[1]{\lVert #1\rVert}

% Entornos
\theoremstyle{definition}
\newtheorem{theorem}[subsection]{Teorema}
\newtheorem{proposition}[subsection]{Proposición}
\newtheorem{lemma}[subsection]{Lema}
\newtheorem{corollary}[subsection]{Corolario}
\newtheorem{observation}[subsection]{Observación}

\newenvironment{ejemplo}{\noindent\textbf{Ejemplo: }}{}
\newenvironment{demostracion}{\noindent\textit{Demostración: } $ $ \newline}{\newline\qed}

\title{\textbf{Cuestiones práctica 2\\ {\small Ampliación de análisis numérico}}}
\date{}
\author{}

\begin{document}
	\maketitle
	
	\noindent\textbf{Cuestión 1:} En la instrucción \texttt{G = chol(A, 'lower')} el parámetro \texttt{'lower'} indica a la función que utilice la parte triangular inferior de la matriz  y la diagonal de $A$ para calcular la factorización de Cholesky. Si $A = R R^t$ al suprimir el parámetro \texttt{'lower'} se obtiene que la matriz $R$ es triangular superior. \\

	\noindent\textbf{Cuestión 2:} Si $n\in\NN$ es la dimensión de la matriz, $\bm{a}\in{\RR}^n$ el vector con la diagonal principal y $\bm{b}\in{\RR}^{n-1}$ el vector con los elementos de la primera subdiagonal principal superior e inferior:
	\begin{enumerate}
		\item \texttt{A=spdiags(a,0,n,n)} crea una matriz dispersa de tamaño $n\times n$ con $\bm{a}$ en la diagonal principal.
		\item \texttt{A=spdiags(b,-1,A)} reemplaza en la matriz $A$ la diagonal inferior a la diagonal principal por el vector $\bm{b}$.
		\item \texttt{A=spdiags([1 b']', 1, A)} esencialmente hace lo mismo, substituir en $A$ la diagonal superior a la principal por el vector $\bm{b}$.
	
	\end{enumerate}
	
	La dos últimas instrucciones son diferentes cuando hacen la misma tarea, esto se debe a que la función se ejecuta con argumentos \texttt{spdiags(B,d,A)} con $A$ la matriz donde queremos subsituir las diagonales en las posiciones $\bm{d}\in{\RR}^k$ por las columnas de $B$. Para que los argumentos sean correctos, si $A\in\mathcal{M}_{n, m}(\RR)$, la matriz $B$ tiene que tener dimensión $\min\{n, m\}\times k$ \\
		
	\noindent\textbf{Cuestión 3:} Los tiempos obtenidos son muy diferentes, si el programa sabe que las matrices son dispersas el tiempo no sube de los $0.1$ segundos, en el caso contrario para dimensión $n=16384$ los tiempos de la factorización $LU$ y Cholesky son $12.85$ segundos y $4.32$ segundos respectivamente.
	
	La variable \textit{nexp} almacena el número de esperimentos a realizar. Cuando $ind\_disp=2$ el número de experimentos es mayor para demostrar que indicando que las matrices son dispersas los algoritmos son mucho más rápidos incluso con matrices más grandes.
	
\end{document}